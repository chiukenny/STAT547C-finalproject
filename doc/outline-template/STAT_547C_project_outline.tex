%%%%%%%%%%%%%%%%%%%%%%%%%%%%%%%%%%%%%%%%%%%%%%%%%%%%%%%%%%%%%%%%%%%%%%%%%%%%%%%%%%%%
% Template for STAT 547C Final Project Outline
% Author: Ben Bloem-Reddy <benbr@stat.ubc.ca>
% Date: Oct. 17, 2019
% Acknowledgments: ETH, Peter Orbanz, John Cunningham
%%%%%%%%%%%%%%%%%%%%%%%%%%%%%%%%%%%%%%%%%%%%%%%%%%%%%%%%%%%%%%%%%%%%%%%%%%%%%%%%%%%%

\documentclass[]{STAT_547C}
\usepackage{STAT_547C}
% NOTE: change the name and email address to your name in STAT_547C.sty

\usepackage{booktabs}
\usepackage{amsmath,amsthm,amssymb,amsfonts}

\usepackage[sorting=none,backend=biber,bibstyle=alphabetic,citestyle=alphabetic,giveninits=true,natbib=true]{biblatex}
\bibliography{../../ref/STAT_547C.bib} % add the title and location of your bibliography file

\begin{document}

% NOTE: You will replace the title below with your actual Title.
\makeGenericHeader{Automating Detection and Breaking of Parameter Symmetries in\\Probabilistic Programs}{Project Outline}
\vspace{-2cm}


%%%%%%%%%%%%%%%%%%%
\section{Title}

The working title of my project is \emph{Automating Detection and Breaking of Parameter Symmetries in\\Probabilistic Programs}.  



%%%%%%%%%%%%%%%%%%%
\section{Background}

Probabilistic programming is a modern programming paradigm in which inference is automatically performed on probabilistic models that are specified by the user. By abstracting away the inference step, users have the flexibility to work with multiple models and leave the choice of inference algorithm to the underlying inference engine. However, existing inference algorithms do not work well in all models, with one challenge being the possible presence of symmetries in the parameterization of a model. Consequences of having parameter symmetries include poor interpretability of nonidentifiable parameters and reduced inference performance. The paper by Nishihara et al. \cite{Nishihara:2013} introduces methods for automatically detecting various types of parameter symmetries in probabilistic programs. I am interested in implementing their algorithms for detecting symmetries in a probabilistic programming language, and taking steps towards an automatic symmetry breaker for the detectable symmetries. This will include a literature review of symmetry breaking solutions that have been proposed for specific probabilistic models.

%%%%%%%%%%%%%%%%%%%
\section{Technical aspects}

The project will draw on technical aspects of the following areas: probabilistic programming, conditioning, graphical models, symmetry analysis.


%%%%%%%%%%%%%%%%%%%
\section{Literature}

The key references for this project are:

\begin{itemize}
\item \cite{Nishihara:2013}, as mentioned above.
\item \cite{Rainforth:2017} is a comprehensive reference for the foundations of probabilistic programming.
\item \cite{Meent:2018} is an introductory reference for probabilistic programming with more emphasis on syntax and model specification than \cite{Rainforth:2017}.
\item Based on a preliminary literature search, some references for symmetry breaking solutions for specific models:
\begin{itemize}
\item Translational, scaling, and sign-flip symmetry in the logistic item-response model \cite{Bafumi:2005}.
\item Translational and scaling symmetry in the multinomial probit model \cite{Nobile:1998}.
\item Permutation symmetry in mixture models \cite{Stephens:2000}.
\item Sign-flip and translational symmetry in the $k$-factor model \cite{Lopes:2004,Erosheva:2017}.
\end{itemize}
\end{itemize}


%%%%%%%%%%%%%%%%%%%
\section{Plan}

I will carry out this project with the following sequence of steps: 
\begin{enumerate}
\item I will implement the symmetry-detecting algorithms presented in the paper by Nishihara et al. \cite{Nishihara:2013} in a probabilistic programming language such as \verb|Infer.NET|.
\item I will search for and review symmetry breaking solutions proposed for specific models.
\item I will focus on one or more of these solutions and automate them in the general context that a certain type of symmetry has been detected in a model.
\end{enumerate}


%%%%%%%%%%%%%%%%%%%
\section{Why I'm interested in this topic}

I have attended a guest-lecture on probabilistic programming by Frank Wood and I found the topic interesting. Doing a project in the context of probabilistic programming seems like a great way to learn more about the programming paradigm and its uses.


%%%%%%%%%%%%%%%%%%%
\printbibliography


\end{document}

