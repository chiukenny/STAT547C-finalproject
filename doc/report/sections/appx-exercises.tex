% !TEX root = ../main.tex

% Exercises section

\section{Exercises}

\begin{enumerate}

\item
This exercise is a worked extended example from Nishihara et al. \cite{Nishihara:2013}. The factor graph from Figure \ref{fig:factorgraph} is shown again here.

\begin{figure}[h]
\begin{center}
\tikz{
\node[latent, xshift=-1cm] (t1) {$\theta_1$};
\node[latent, xshift=1cm] (t2) {$\theta_2$};
\node[latent, yshift=-2cm] (t3) {$\theta_3$};
\node[latent, xshift=2cm, yshift=-2cm] (t4) {$\theta_4$};
\node[obs, xshift=1cm, yshift=-4cm] (t5) {$\theta_5$};
\factor[above=of t1] {f1} {above:Gaussian} {} {t1};
\factor[above=of t2] {f2} {above:Gaussian} {} {t2};
\factor[below=of t1, xshift=1cm] {fplus} {left:sum} {t1,t2} {t3};
\factor[above=of t4] {f4} {above:Gaussian} {} {t4};
\factor[below=of t3, xshift=1cm] {fprod} {left:product} {t3,t4} {t5}
}
\end{center}
\end{figure}

Some factors and the scaling constraints that they impose is provided in the following table.

\begin{table}[h]
\centering
\begin{tabular}{|c|c|}
\hline
factor & constraints \\
\hline
$c=a+b$ & $d_a = d_b = d_c$ \\
$c = a\times b$ & $d_c= d_a + d_b$ \\
$c$ observed & $d_c=0$ \\
\hline
\end{tabular}
\end{table}

\begin{enumerate}

\item
Find the scaling constraint matrix $\mathcal{C}$ for this factor graph.

\begin{solution}
The sum factor adds the rows $d_1 - d_3 = 0$ and $d_2 - d_3 = 0$. \\
The product factor adds the row $d_3 + d_4 - d_5 = 0$. \\
The observed factor adds the row $d_5 = 0$.

Aggregating the rows produces the following constraint matrix.
\[
\mathcal{C} =
\begin{bmatrix}
1 & 0 & -1 & 0 & 0 \\
0 & 1 & -1 & 0 & 0 \\
0 & 0 & 1 & 1 & -1 \\
0 & 0 & 0 & 0 & 1
\end{bmatrix}
\]
\end{solution}

\item
Find the null space of $\mathcal{C}$.

\begin{solution}
A basis for the null space of $\mathcal{C}$ is given by the solution to the following system of equations.
\[
\begin{bmatrix}
1 & 0 & -1 & 0 & 0 \\
0 & 1 & -1 & 0 & 0 \\
0 & 0 & 1 & 1 & -1 \\
0 & 0 & 0 & 0 & 1
\end{bmatrix}
\begin{bmatrix}
d_1 \\ d_2 \\ d_3 \\ d_4 \\ d_5
\end{bmatrix}
=
\begin{bmatrix}
0 \\ 0 \\ 0 \\ 0 \\ 0
\end{bmatrix}
\]
Solving for the solution leads to the null space
\[
\mathcal{N}(\mathcal{C}) = \left\{c
\begin{bmatrix}
1 \\ 1 \\ 1 \\ -1 \\ 0
\end{bmatrix}
: c\in\mathbb{R}
\right\}
\]
\end{solution}

\item
Under the equivalence class formulation, give the definition of the scaling symmetry detector for this factor graph $\mathcal{F}$.

\begin{solution}
The scaling symmetry detector is defined as
\[
\Delta_\mathcal{F}(\theta) = \{\Sigma_d\theta:d\in\mathcal{N}(\mathcal{C})\}
\]
where $\Sigma_d$ is the diagonal matrix with $(e^{d_1},...,e^{d_5})$ on the diagonal and $\mathcal{N}(\mathcal{C})$ is as found in the previous question. 
\end{solution}

\item
Given $\theta^*=(\theta_1^*,...,\theta_5^*)^T$, find $[\theta^*]$.

\begin{solution}
The equivalence class of $\theta^*$ can be found using the symmetry detector defined in the previous question.
\[
[\theta^*] = \Delta_\mathcal{F}(\theta^*) = \{\Sigma_d\theta^*:d\in\mathcal{N}(\mathcal{C})\} = \left\{
\begin{bmatrix}
e^c\theta_1^* \\
e^c\theta_2^* \\
e^c\theta_3^* \\
e^{-c}\theta_4^* \\
\theta_5^*
\end{bmatrix}
:
c\in\mathbb{R}
\right\}
\]
\end{solution}

\item
Let $\|\cdot\|$ denote the Euclidean norm. Suppose that for any equivalence class $[\theta]$ of $\mathcal{F}$, no two members have the same norm. For some $r\in\mathbb{R}_+$, consider a function of the form
\[
\phi_\mathcal{F}([\theta])=\{\theta:\|\theta\|=r,\theta\in[\theta]\}
\]
Explain why this function is a not a well-defined symmetry breaker for all $[\theta]$. (\textit{Remark: here we use a singleton to denote the representative to allow this $\phi_\mathcal{F}$ to be defined in simple notation.})

\begin{solution}
By our definition, a symmetry breaker must return a member of the given equivalence class. It is possible that no member $\theta$ of the equivalence class satisfies $\|\theta\|=r$. For example, $[\theta^*]$ in the previous question has no members with $\|\theta\|<|\theta^*_5|$. Hence $\phi_\mathcal{F}$ is not a valid symmetry breaker for all equivalence classes in $\Theta/\Xi$.

\textit{Remark: using this factor graph as an example, $\theta_5$ is assumed to be fixed at some value $\theta_5^*$ and so any $\theta'$ with $\theta_5'\neq\theta_5^*$ has $\mathbb{P}(\theta'=0)$. Symmetry breakers that are valid only over non-null sets may still be useful but are invalid under our current definition. This may suggest that the definition of symmetry breakers should be reconsidered to view $[\theta]$ as random with a probability distribution.}
\end{solution}

\end{enumerate}

\item
Consider the equivalence class formulation. Show why a permutation symmetry breaker that selects a representative based on its Euclidean norm would not be useful.

\begin{solution}
Let $[\theta]$ be any permutation equivalence class. Norm-based symmetry breakers are not useful for permutation symmetries because all members of an equivalence class have the same norm. By definition, if $\theta$, $\theta^*\in[\theta]$ then $\theta^*=\Pi^*\theta$ for some $\Pi^*\in\mathcal{F}$. Then
\[
\|\theta^*\| = \|\Pi^*\theta\| = \|\theta\|
\]
because $\Pi^*$ is orthogonal.
\end{solution}

\newpage

\item
Consider the following factor graph that has an exponential prior with rate $\beta = \frac{1}{2}$ and a chi-squared prior with degrees of freedom $d=2$:

\begin{figure}[h]
\begin{center}
\tikz{
\node[latent, xshift=-1cm] (t1) {$\theta_1$};
\node[latent, xshift=1cm] (t2) {$\theta_2$};
\node[latent, yshift=-2cm] (other) {...};
\factor[above=of t1] {f1} {above:Exp$\left(\frac{1}{2}\right)$} {} {t1};
\factor[above=of t2] {f2} {above:$\chi^2(2)$} {} {t2};
\factor[below=of t1,xshift=1cm] {fplus} {left:sum} {t1,t2} {other}
}
\end{center}
\end{figure}

This factor graph has a permutation symmetry. Explain why and suggest one way to break the symmetry under the reparameterization formulation.

Let $\lambda$ denote the Lebesgue measure. You may find the following useful:
\[
\mathbb{P}(\theta_1\in dx) = \lambda(dx)\beta e^{-\beta x} \qquad x\in\mathbb{R}_+
\]
\[
\mathbb{P}(\theta_2\in dx) = \lambda(dx)\frac{1}{2^{\frac{d}{2}}\Gamma\left(\frac{d}{2}\right)}x^{\frac{d}{2}-1}e^{-\frac{x}{2}} \qquad x\in\mathbb{R}_+
\]

\begin{solution}
Notice that
\[
\mathbb{P}(\theta_1\in dx) = \lambda(dx)\frac{1}{2}e^{-\frac{1}{2}x} \qquad x\in\mathbb{R}_+
\]
and
\[
\mathbb{P}(\theta_2\in dx) = \lambda(dx)\frac{1}{2^{\frac{2}{2}}\Gamma\left(\frac{2}{2}\right)}x^{\frac{2}{2}-1}e^{-\frac{x}{2}} = \lambda(dx)\frac{1}{2}e^{-\frac{x}{2}} \qquad x\in\mathbb{R}_+
\]
and so $\theta_1$ and $\theta_2$ have the same distribution. Hence the permutation symmetry exists because the inputs to the sum factor are symmetric. In this case, the symmetry can be broken by merging the sum factor with its inputs as the sum is specified by a gamma distribution with shape $\alpha=2$ and rate $\beta=\frac{1}{2}$. We show this using the Laplace transform. For $r\in\mathbb{R}_+$, we have
\[
\mathbb{E}[e^{-r\theta_1}] = \mathbb{E}[e^{-r\theta_2}] = \frac{2^{-1}}{2^{-1}+r}
\]
Then by independence of $\theta_1$ and $\theta_2$,
\[
\mathbb{E}[e^{-r(\theta_1+\theta_2)}] = \mathbb{E}[e^{-r\theta_1}]\mathbb{E}[e^{-r\theta_2}] = \left(\frac{2^{-1}}{2^{-1}+r}\right)^2
\]
which corresponds to the Laplace transform of the mentioned gamma distribution. The new factor graph obtained from merging is

\begin{figure}[h]
\begin{center}
\tikz{
\node[latent] (t12) {$\theta_{12}$};
\node[latent, yshift=-1.25cm] (other) {...};
\factor[above=of t12] {f1} {above:Gamma$\left(2,2^{-1}\right)$} {} {t12};
\edge {t12} {other}
}
\end{center}
\end{figure}
\end{solution}

\end{enumerate}