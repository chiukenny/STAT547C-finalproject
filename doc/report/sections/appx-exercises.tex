% !TEX root = ../main.tex

% Exercises section

\section{Exercises}

\begin{enumerate}

\item
This exercise is a worked extended example from Nishihara et al. \cite{Nishihara:2013}. The factor graph from Figure \ref{fig:factorgraph} is shown again here.

\begin{figure}[h]
\begin{center}
\tikz{
\node[latent, xshift=-1cm] (t1) {$\theta_1$};
\node[latent, xshift=1cm] (t2) {$\theta_2$};
\node[latent, yshift=-2cm] (t3) {$\theta_3$};
\node[latent, xshift=2cm, yshift=-2cm] (t4) {$\theta_4$};
\node[obs, xshift=1cm, yshift=-4cm] (t5) {$\theta_5$};
\factor[above=of t1] {f1} {above:Gaussian} {} {t1};
\factor[above=of t2] {f2} {above:Gaussian} {} {t2};
\factor[below=of t1, xshift=1cm] {fplus} {left:sum} {t1,t2} {t3};
\factor[above=of t4] {f4} {above:Gaussian} {} {t4};
\factor[below=of t3, xshift=1cm] {fprod} {left:product} {t3,t4} {t5}
}
\end{center}
\end{figure}

Some factors and the scaling constraints that they impose is provided in the following table.

\begin{table}[h]
\centering
\begin{tabular}{|c|c|}
\hline
factor & constraints \\
\hline
$c=a+b$ & $d_a = d_b = d_c$ \\
$c = a\times b$ & $d_c= d_a + d_b$ \\
$c$ observed & $d_c=0$ \\
\hline
\end{tabular}
\end{table}

\begin{enumerate}

\item
Find the scaling constraint matrix $\mathcal{C}$ for this factor graph.

\begin{solution}
The sum factor adds the rows $d_1 - d_3 = 0$ and $d_2 - d_3 = 0$. \\
The product factor adds the row $d_3 + d_4 - d_5 = 0$. \\
The observed factor adds the row $d_5 = 0$.

Aggregating the rows produces the following constraint matrix.
\[
\mathcal{C} =
\begin{bmatrix}
1 & 0 & -1 & 0 & 0 \\
0 & 1 & -1 & 0 & 0 \\
0 & 0 & 1 & 1 & -1 \\
0 & 0 & 0 & 0 & 1
\end{bmatrix}
\]
\end{solution}

\item
Find the null space of $\mathcal{C}$.

\begin{solution}
The null space of $\mathcal{C}$ is the solution to the following system of equations:
\[
\begin{bmatrix}
1 & 0 & -1 & 0 & 0 \\
0 & 1 & -1 & 0 & 0 \\
0 & 0 & 1 & 1 & -1 \\
0 & 0 & 0 & 0 & 1
\end{bmatrix}
\begin{bmatrix}
d_1 \\ d_2 \\ d_3 \\ d_4 \\ d_5
\end{bmatrix}
=
\begin{bmatrix}
0 \\ 0 \\ 0 \\ 0 \\ 0
\end{bmatrix}
\]
Solving for the solution gives the null space
\[
\mathcal{N}(\mathcal{C}) = \left\{c
\begin{bmatrix}
1 \\ 1 \\ 1 \\ -1 \\ 0
\end{bmatrix}
: c\in\mathbb{R}
\right\}
\]
\end{solution}

\item
Under the equivalence class formulation, give the definition of the scaling symmetry detector for this factor graph $\mathcal{F}=(\theta,F,\Xi)$ where $\Xi$ is the equivalence relation for the scaling symmetry.

\begin{solution}
The scaling symmetry detector is defined as
\[
\Delta_\mathcal{F}(\theta) = \{\Sigma_d\theta:d\in\mathcal{N}(\mathcal{C})\}
\]
where $\Sigma_d$ is the diagonal matrix with $(e^{d_1},...,e^{d_5})$ on the diagonal and where $\mathcal{N}(\mathcal{C})$ is as found in the previous question. 
\end{solution}

\item
Given $\theta^*=(\theta_1^*,...,\theta_5^*)^T$, what is $[\theta^*]$?

\begin{solution}
The equivalence class of $\theta^*$ can be found using the symmetry detector defined in the previous question.
\[
[\theta^*] = \Delta_\mathcal{F}(\theta^*) = \{\Sigma_d\theta^*:d\in\mathcal{N}(\mathcal{C})\} = \left\{
\begin{bmatrix}
e^{d_1}\theta_1^* \\
\vdots \\
e^{d_5}\theta_5^*
\end{bmatrix}
:
d\in\mathcal{N}(\mathcal{C})
\right\}
\]
\end{solution}

\item
Let $\|\cdot\|$ be the Euclidean norm. Define
\[
\phi_\mathcal{F}([\theta])=\underset{\theta\in[\theta],\|\theta\|=1}{\arg\min}\;\sum_{n=1}^5\theta_n
\]
Is $\phi_\mathcal{F}$ a scaling symmetry breaker?

\begin{solution}
No, $\phi_\mathcal{F}$ is not a scaling symmetry breaker as it does not necessarily return a single member of $[\theta]$.

To see this, suppose that $\theta^*=(\theta_1^*,...,\theta_5^*)^*\in[\theta^*]$ satisfies $\|\theta^*\|=1$ and the minimum sum of components. Then $\theta'=(\theta_2^*,\theta_1^*,\theta_3^*,\theta_4^*,\theta_5^*)^T$ is also in $[\theta^*]$, satisfies $\|\theta'\|=1$ and also has the minimum sum of components. Hence $\phi_\mathcal{F}([\theta^*])$ returns at least two members of $[\theta^*]$.

TODO
\end{solution}

\end{enumerate}

\end{enumerate}