% !TEX root = ../main.tex

% Body 2 section

\section{Formulation based on reparameterization}

The formulation based on equivalence classes is useful when the main concern of having symmetries is parameter nonidentifiability. In this case, it is assumed that the inference algorithm is able to run as normal and that the symmetries can be broken post-inference. However, there may also be cases where the presence of symmetries impact the performance of the inference algorithm. In this situation, the symmetry should be broken before running inference. We present a formulation of symmetry breaking based on reparameterizing the model which is done pre-inference. The formulation directly builds on the symmetry detection approaches proposed in Nishihara et al. by changing the factor graph in such a way that the approaches detect no symmetries.


\subsection{Scaling symmetries}

The approach proposed by Nishihara et al. for detecting scaling symmetries is based on constructing and finding the null space to the constraint matrix $\mathcal{C}$. In other words, a scaling symmetry is present if the constraint matrix is underdetermined. We look to change the factor graph such that the constraint matrix becomes a determined system. We present this approach from two perspectives. 

The first perspective makes the constraint matrix determined by deleting columns. This corresponds to reducing the number of variables in the factor graph by either removing or merging factors. How this can be done will largely depend on the factor graph and the source of the scaling symmetry. For example, the ternary sum of common distributions from a single distribution family is typically specified by another well-known distribution. If a scaling symmetry is introduced through a sum factor that takes in multiple Gaussian inputs, the model can be reparameterized such that the sum and the Gaussian inputs are replaced by a single Gaussian factor representing the sum. This breaks the scaling symmetry originating from the original sum factor in the factor graph.

Rather than making the constraint matrix smaller, the second perspective adds new rows to the constraint matrix. This corresponds to imposing new constraints through additional factors. TODO


\subsection{Permutation symmetries}

Nishihara et al. reduces detection of permutation symmetries to a problem of finding the automorphism of a labeled graph. An automorphism may exist due to the symmetric arguments of a factor being annotated with the same label. The symmetry in the arguments can be removed however by imposing an additional constraint based on ordering. For example, the sum factor $c=a+b$ is no longer symmetric with respect to $a$ and $b$ under the constraint $a<b$. This is equivalent to reparameterizing the factor graph such that the inputs to the factor are order statistics. Under this new parameterization, the inputs no longer receive the same label and so the only automorphism of the factor graph is the factor graph itself.


% ...