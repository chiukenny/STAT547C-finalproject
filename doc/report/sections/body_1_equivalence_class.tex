% !TEX root = ../main.tex

% Body 1 section

\section{Formulation based on equivalence classes}

Let $(\theta,F)$ be a factor graph. Let $\Xi$ be an equivalence relation such that for $\theta,\theta_*\in\Theta$, we have $\theta\Xequiv\theta^*$ if and only if there exists a local symmetry $\sigma$ such that $\sigma(\theta)=\theta^*$.

We show that $\Xi$ is a proper equivalence relation. Consider $\theta_1, \theta_2, \theta_3\in\Theta$.
\begin{enumerate}

\item
\textbf{reflexivity}: let $\sigma$ be the identity map. Then trivially, $F_k(\theta_1) = F_k(\sigma(\theta_1))$.

\item
\textbf{symmetry}: suppose $\theta_1\Xequiv\theta_2$. Then there exists some local symmetry $\sigma$ such that $\sigma(\theta_1)=\theta_2$ and $F_k(\theta_1) \propto F_k(\theta_2)$. By definition of symmety, $\sigma^{-1}$ exists and $\sigma^{-1}(\theta_2)=\theta_1$. Then
\[
F_k(\theta_2) \propto F_k(\theta_1) = F_k(\sigma^{-1}(\theta_2))
\]
and hence $\theta_2\Xequiv\theta_1$.

\item
\textbf{transitivity}: suppose $\theta_1\Xequiv\theta_2$ and $\theta_2\Xequiv\theta_3$. Then there exists local symmetries $\sigma_1$, $\sigma_2$ such that
\begin{align*}
\sigma_1(\theta_1)&=\theta_2 & F_k(\theta_1)&\propto F_k(\theta_2) \\
\sigma_2(\theta_2)&=\theta_3 & F_k(\theta_2)&\propto F_k(\theta_3)
\end{align*}
Let $\sigma_3=\sigma_2\circ\sigma_1$ and so $\sigma_3$ is also measurable with a measurable inverse. Then
\[
\sigma_3(\theta_1)=\sigma_2(\sigma_1(\theta_1)) = \theta_3
\]
and
\[
F_k(\theta_1) \propto F_k(\theta_2) \propto F_k(\theta_3) = F_k(\sigma_3(\theta_1))
\]
and hence $\theta_1\Xequiv\theta_3$.

\end{enumerate}

Notice that $\Xi$ is only describable in the context of a given factor graph $(\theta,F)$. It is therefore mathematically more convenient to consider factor graphs equipped with a $\Xi$ that corresponds to a specific type of symmetry. We denote this as $\mathcal{F}=(\theta,F,\Xi)$. The constraint matrix $\mathcal{C}$ of $\mathcal{F}$ described in Section \ref{sec:scaling} is constructed from the constraints specified by $\Xi$ on each factor $F_k$.

We now have all the definitions needed to formulate symmetry detection and symmetry breaking in terms of equivalence classes.
\begin{itemize}

\item
A symmetry detector $\Delta_\mathcal{F}:\Theta\rightarrow\Theta/\Xi$ is a measurable function that maps $\theta$ to $[\theta]$ with respect to $\Xi$.

\item
A symmetry breaker $\phi_\mathcal{F}:\Theta/\Xi\rightarrow\Theta$ is a measurable section that maps $[\theta]$ to a representative $\theta_*\in[\theta]$.

\item
An automatic symmetry breaker $\phi_\mathcal{F}\circ\Delta_\mathcal{F}=\Phi_\mathcal{F}:\Theta\rightarrow\Theta$ is the composition of a symmetry breaker and its corresponding symmetry detector. It maps $\theta$ to a representative $\theta_*$ of its equivalence class.

\end{itemize}

The dependence of $\Delta$, $\phi$, and $\Phi$ on a factor graph $\mathcal{F}$ is made clear by the subscript. When considering two functions of the same type, e.g. $\Phi_{\mathcal{F}_1}$ and $\Phi_{\mathcal{F}_2}$, we will view them as being defined on the same factor graph $(\theta,F)$ but with $\Xi_1,$ and $\Xi_2$ corresponding to different types of symmetries.

This formulation is most useful in the case where the main concern is the nonidentifiability of the parameters in the presence of symmetries. If inference has been made on a posterior distribution but the question remains whether the inferred value is the ``correct" one, it may instead be more interpretable to work with the symmetry broken posterior distribution
\[
\prod_kF_k(\Phi_{\mathcal{F}_M}\circ ...\circ\Phi_{\mathcal{F}_1}(\theta))
\]
where $\Phi_{\mathcal{F}_1},...,\Phi_{\mathcal{F}_M}$ correspond to $M$ different types of symmetries that are to be broken. The corresponding $\phi_{\mathcal{F}_1},...,\phi_{\mathcal{F}_M}$ are chosen to select representatives from the respective equivalence classes based on some desired criteria.

We show an example to demonstrate this idea.


\subsection{Scaling symmetries}

We describe the scaling symmetries in Section \ref{sec:scaling} under the formulation based on equivalence classes. Let $\Sigma_d$ be the diagonal matrix with $(e^{d_1},...,e^{d_N})$ on the diagonal. Then a scaling symmetry $\sigma_d$ can be written as
\[
\sigma_d(\theta)=\Sigma_d\theta
\]

We define the scaling symmetry detector as
\[
\Delta_\mathcal{F}(\theta) = \{\Sigma_d\theta:d\in\mathcal{N}(\mathcal{C})\}=[\theta]
\]
where $\mathcal{C}$ is the constraint matrix of $\mathcal{F}$. We show that this detector correctly maps the variables $\theta$ to their equivalence classes $[\theta]$ induced by $\Xi$. That is, the detector maps two variables $\theta$, $\theta^*\in\Theta$ to the same equivalence class if and only if there exists a scaling symmetry $\sigma$ such that $\sigma(\theta)=\theta^*$.

Suppose that $\theta^*=\Sigma_{d^*}\theta$ for some $d^*\in\mathcal{N}(\mathcal{C})$. Then
\begin{align*}
\Delta_\mathcal{F}(\theta^*) &= \{\Sigma_d\theta^*:d\in\mathcal{N}(\mathcal{C})\} \\
&= \{\Sigma_d\Sigma_{d^*}\theta:d\in\mathcal{N}(\mathcal{C})\} \\
&= \{\Sigma_{d+d^*}\theta:d\in\mathcal{N}(\mathcal{C})\} \\
&= \{\Sigma_d\theta:d\in\mathcal{N}(\mathcal{C})\} \\
&= \Delta_\mathcal{F}(\theta)
\end{align*}
where the above follows because $\Sigma_d$ and $\Sigma_{d^*}$ are diagonal and $d+d^*\in\mathcal{N}(\mathcal{C})$.

We show the converse by the contrapositive. Suppose that $\theta^*\neq\Sigma_{d}\theta$ for all $d\in\mathcal{N}(\mathcal{C})$. Then $\theta^*\notin\Delta_\mathcal{F}(\theta)$ by definition. Also, by definition of equivalence class, $\theta^*\in[\theta^*]=\Delta_\mathcal{F}(\theta^*)$. Hence $\Delta_\mathcal{F}(\theta)\neq\Delta_\mathcal{F}(\theta^*)$ and so the detector maps $\theta$, $\theta^*$ to different equivalence classes.

A scaling symmetry breaker $\phi_\mathcal{F}$ is more difficult to formalize for two reasons. The first reason is that the equivalence class is described by using the input $\theta$ as a reference point. If $[\theta_1]$ and $[\theta_2]$ describe the same equivalence class, the breaker must be able to map both to the representative $\theta^*$. The second reason is that the equivalence class is nonlinear in the space of diagonal matrices $\Sigma_d$ (it is described by the linear space $\mathcal{N}(\mathcal{C})$). This means that for a given $\theta$, operations such as scaling may not return a member of the same equivalence class. These two reasons make it challenging to correctly identify the representative for the given equivalence class. An example of a possible breaker worth exploring is the one that returns the minimum norm, i.e.,
\[
\phi_\mathcal{F}([\theta]) = \underset{\theta\in[\theta]}{\arg\min}\;\|\theta\|
\]
However, consideration will need to be given in how to deal with multiple members having the same minimum norm and also in how this could be solved computationally for an uncountable $[\theta]$.

Under the assumption that the representative $\theta^*$ is known for all descriptions $[\cdot]$ of any equivalence class in $\Theta/\Xi$, it is straightforward to transform any $\theta$ to the representative of its class. Note that
\[
\theta = \Sigma_{d^*}\theta^*
\]
for some $d^*\in\mathcal{N}(\mathcal{C})$. Then because $\Sigma_{d^*}$ is positive definite, the inverse exists and hence
\[
\Sigma_{d^*}^{-1}\theta = \theta^*
\]


\subsection{Permutation symmetries}

While not a local symmetry, the permutation symmetry discussed in Section \ref{sec:permutation} can also be formulated in terms of equivalence classes. In this case, the definition of $\Xi$ is modified to consider the existence of a non-local symmetry between two members of an equivalence class. Let $\Pi$ be a permutation matrix (a matrix where rows of the identity matrix are permuted). Then a permutation symmetry $\sigma$ can be written as
\[
\sigma(\theta)=\Pi\theta
\]
We say that $\Pi\in\mathcal{F}$ if the rows that correspond to non-permutable variables have 1 on the diagonal. The permutation symmetry detector can then be defined as
\[
\Delta_\mathcal{F}(\theta) = \left\{\Pi\theta:\Pi\in\mathcal{F}\right\} = [\theta]
\]
Note that if $\Pi,\Pi^*\in\mathcal{F}$, then the matrix product $\Pi\Pi^*$ is itself a permutation matrix by properties of permutation matrices. Furthermore, $\Pi\Pi^*\in\mathcal{F}$ as both matrices have 1 on the diagonal for non-permutable rows and thus so does the product.

This fact directly shows that $\Delta_\mathcal{F}(\theta)=\Delta_\mathcal{F}(\theta^*)$ if there exists a permutation symmetry $\sigma$ such that $\sigma(\theta)=\Pi^*\theta=\theta^*$, $\Pi^*\in\mathcal{F}$, as
\begin{align*}
\Delta_\mathcal{F}(\theta^*) &= \left\{\Pi\theta^*:\Pi\in\mathcal{F}\right\} \\
&= \left\{\Pi\Pi^*\theta:\Pi\in\mathcal{F}\right\} \\
&= \left\{\Pi\theta:\Pi\in\mathcal{F}\right\} \\
&= \Delta_\mathcal{F}(\theta)
\end{align*}
The proof for the converse follows the same argument for the scaling symmetry. Hence this permutation symmetry detector correctly maps the variables to their equivalent classes.

An example of a permutation symmetry breaker is easy to formulate. For convenience, suppose that the permutable variables in $\theta$ have distinct values. Then an example permutation breaker $\phi_\mathcal{F}$ is one such that $\phi_\mathcal{F}([\theta])=\theta^*$ where if $\theta_i$ and $\theta_j$ are permutable, then $\theta^*_i<\theta^*_j$. This breaker is well-defined as there are only at most $N!$ possible permutations of $\theta$. In the context of a probabilistic programming, this reduces to applying a sorting algorithm to certain entries of $\theta$.


% ...