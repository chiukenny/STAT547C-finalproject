% !TEX root = ../main.tex

% Body 1 section

\section{Formulation based on equivalence classes}

Let $(\theta,F)$ be a factor graph. Let $\Xi$ be an equivalence relation such that for $\theta,\theta_*\in\Theta$, we have $\theta\Xequiv\theta_*$ if and only if there exists a local symmetry $\sigma$ such that $\sigma(\theta)=\theta_*$.

We show that $\Xi$ is a proper equivalence relation. Consider $\theta_1, \theta_2, \theta_3\in\Theta$.
\begin{enumerate}

\item
\textbf{reflexivity}: let $\sigma$ be the identity map. Then trivially, $F_k(\theta_1) = F_k(\sigma(\theta_1))$.

\item
\textbf{symmetry}: suppose $\theta_1\Xequiv\theta_2$. Then there exists some symmetry $\sigma$ such that $\sigma(\theta_1)=\theta_2$ and $F_k(\theta_1) \propto F_k(\theta_2)$. By definition of symmety, $\sigma^{-1}$ exists and $\sigma^{-1}(\theta_2)=\theta_1$. Then
\[
F_k(\theta_2) \propto F_k(\theta_1) = F_k(\sigma^{-1}(\theta_2))
\]
and hence $\theta_2\Xequiv\theta_1$.

\item
\textbf{transitivity}: suppose $\theta_1\Xequiv\theta_2$ and $\theta_2\Xequiv\theta_3$. Then there exists symmetries $\sigma_1$, $\sigma_2$ such that
\begin{align*}
\sigma_1(\theta_1)&=\theta_2 & F_k(\theta_1)&\propto F_k(\theta_2) \\
\sigma_2(\theta_2)&=\theta_3 & F_k(\theta_2)&\propto F_k(\theta_3)
\end{align*}
Then $\sigma_2(\sigma_1(\theta_1)) = \theta_3$ and
\[
F_k(\theta_1) \propto F_k(\theta_2) \propto F_k(\theta_3) = F_k(\sigma_2(\sigma_1(\theta_1)))
\]
and hence $\theta_1\Xequiv\theta_3$.

\end{enumerate}

Notice that $\Xi$ is dependent on the factor graph $(\theta,F)$. It is mathematically more convenient to consider factor graphs equipped with a $\Xi$ that corresponds to a specific type of symmetry. We denote this as $\mathcal{F}=(\theta,F,\Xi)$.

We now have all the definitions needed to formulate symmetry detection and symmetry breaking in terms of equivalence classes.
\begin{itemize}

\item
A symmetry detector $\Delta_\mathcal{F}:\Theta\rightarrow\Theta/\Xi$ is a measurable function that maps $\theta$ to $[\theta]$ with respect to $\Xi$.

\item
A symmetry breaker $\phi_\mathcal{F}:\Theta/\Xi\rightarrow\Theta$ is a section that maps $[\theta]$ to a representative $\theta_*\in[\theta]$.

\item
An automatic symmetry breaker $\phi_\mathcal{F}\circ\Delta_\mathcal{F}=\Phi_\mathcal{F}:\Theta\rightarrow\Theta$ is the composition of a symmetry breaker and its corresponding symmetry detector. It maps $\theta$ to a representative $\theta_*$ of its equivalence class.

\end{itemize}

The dependence of $\Delta$, $\phi$, and $\Phi$ on a factor graph $\mathcal{F}$ is made clear by its subscript. When considering two functions of the same type, e.g. $\Phi_{\mathcal{F}_1}$ and $\Phi_{\mathcal{F}_2}$, we will view them as being defined on the same factor graph $(\theta,F)$ but with $\Xi_1,$ and $\Xi_2$ corresponding to different types of symmetries.

This formulation is most useful in the case where the main concern is the nonidentifiability of the parameters in the presence of symmetries. If inference has been made on a posterior distribution but the question remains whether the inferred value is the "correct" one, it may instead be more interpretable to work with the symmetry broken posterior distribution
\[
\prod_kF_k(\Phi_{\mathcal{F}_M}\circ ...\circ\Phi_{\mathcal{F}_1}(\theta))
\]
where $\Phi_{\mathcal{F}_1},...,\Phi_{\mathcal{F}_M}$ correspond to $M$ different types of symmetries that are to be broken. The corresponding $\phi_{\mathcal{F}_1},...,\phi_{\mathcal{F}_M}$ are chosen to select representatives from the respective equivalence classes based on some desired criteria.

We show an example to demonstrate this idea.


\subsection{Scaling symmetries}

A scaling symmetry \cite{Nishihara:2013}


% ...