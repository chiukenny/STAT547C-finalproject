% !TEX root = ../main.tex

% open questions section

\section{Open questions and research directions}

Parameter symmetries in models lead to problems of parameter nonidentifiability and reduced inference performance. Although these symmetries can often be dealt with manually as seen in the literature \cite{Bafumi:2005,Stephens:2000}, it is not as easy to handle them automatically in a probabilistic programming context. The paper by Nishihara et al. \cite{Nishihara:2013} presents multiple algorithms for automatically detecting certain types of symmetries. Based on their work, we introduced two mathematical formulations of symmetry detection and breaking that have potential for automation. We conclude this report with a summary of both formulations and their possible research directions.

The equivalence class formulation frames the problem of detection and breaking as mapping the variables to a specific set of variables that they are symmetric to. The formulation is most useful as a post-processing step after inference as an intuitive solution to parameter nonidentifiability. Describing both detection and breaking as mathematical functions also naturally leads to direct translations of functions in a probabilistic programming language. The main challenges in the equivalence class formulation include specifying valid symmetry breakers and how they could be implemented to handle general classes that are nonlinear spaces. There has been some work in formalizing functions defined on equivalence classes \cite{Paulson:2006} that may be of relevance for making progress on these obstacles. Another challenge is the question of how the translation symmetries described by Nishihara et al. could be formulated as functions given their dependence on the input variables. Drawing on results from other areas of mathematics such as group theory may be necessary to solve these problems.

The reparameterization formulation directly works off of the algorithms by Nishihara et al. and aims to break symmetries by changing the factor graph. This formulation allows for symmetries to be broken pre-inference and is appropriate in situations where the performance of the inference algorithm is expected to be affected by the presence of symmetries. However, our formulation of reparameterization highly dependent on the structure and context of the factor graph. This makes it difficult to implement for a general model in its current state. Further development of this formulation is needed to determine how feasible it is to automate this approach. A research direction worth exploring is understanding the class of factor graph transformations that do not change the model or at least allow for the original model to be recovered after inference. Such transformations would mitigate the drawback of implicitly changing the model when the factor graph is changed. The paper by Gelman \cite{Gelman:2004} discusses some reparameterization techniques for improving computability of Bayesian models. We expect that some of these techniques can be repurposed to break parameter symmetries in a model.