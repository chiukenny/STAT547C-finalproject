\documentclass{article}


\usepackage{amsmath}
\usepackage{amssymb}
\usepackage{hyperref}


\begin{document}


\section{Definitions}

\begin{itemize}

\item
We represent probabilistic models using factor graphs, denoted $(\theta,F)$ of variables $\theta=(\theta_1,...,\theta_N)$ and factors $F=(F_1,...,F_k)$.

$\Theta$ denotes the space of variable values. $\theta\in\Theta$ contains all parameters, latent variables, and observations (which are fixed).

$F$ represents all functions, operations, and constraints that the posterior distribution factorizes into from a Bayesian perspective including priors and likelihoods (in directed graphs). Given data, the unnormalized posterior distribution can then be expressed as
\[
\prod_kF_k(\theta)
\]

\item
Symmetry $\sigma:\Theta\rightarrow\Theta$ is a measurable function with a measurable inverse satisfying
\[
\prod_kF_k(\theta)\propto\prod_kF_k\left(\sigma(\theta)\right)
\]
where the product is taken over non-prior terms. Symmetries must also fix the components of $\theta$ that correspond to observed variables.

\item
Local symmetry is a symmetry $\sigma$ that satisfies
\[
F_k(\theta) \propto F_k\left(\sigma(\theta)\right)
\]
for all non-prior factors $F_k$.

\end{itemize}

\subsection{Equivalence class}

\begin{itemize}

\item
A factor graph with equivalence relation $\Xi$, denoted $\mathcal{F}=(\theta,F,\Xi)$, induces a partition of $\Theta$ into equivalence classes $[\cdot]$, with the equivalence relation $\Xi$ that there exists a symmetry between any two members of an equivalence class. $\Xi$ specifies the type of symmetry.

Denote $[\theta]$ to be the equivalence class of $\theta$.

Denote $\Theta/\Xi$ to be the set of all equivalence classes in $\Theta$ w.r.t $\Xi$, called the quotient set of $\Theta$ by $\Xi$.

(TODO: local symmetry?)

\item
(TODO: Specific to breaking symmetry post-inference?)

A symmetry detector $\Delta_\mathcal{F}:\Theta\rightarrow\Theta/\Xi$ is a measurable function that maps $\theta$ to $[\theta]$ w.r.t. $\Xi$.

\item
(TODO: Specific to breaking symmetry post-inference?)

A symmetry breaker $\phi_\mathcal{F}:\Theta/\Xi\rightarrow\Theta$ is a measurable function that maps $[\theta]$ to $\theta_0$ w.r.t. $\Xi$ where $\theta_0$ is a representative of $[\theta]$.

\item
(TODO: Specific to breaking symmetry post-inference?)

An automatic symmetry breaker $\Phi_\mathcal{F}:\Theta\rightarrow\Theta$ is the composition $\Phi_\mathcal{F}=\phi_\mathcal{F}\circ\Delta_\mathcal{F}$ of a symmetry breaker with the corresponding symmetry detector.

(TODO)

An automatic symmetry breaker $\Phi_\mathcal{F}:\Theta\rightarrow\Theta$ is the composition $\Phi_\mathcal{F}=\phi_\mathcal{F}\circ\Delta_\mathcal{F}$ of a symmetry breaker with the corresponding symmetry detector that satisfies the following:
\begin{enumerate}

\item
For ASB for two types of symmetries $\Phi_1$, $\Phi_2$, $\Phi_1\circ\Phi_2(\theta)=\Phi_2\circ\Phi_1(\theta)$

\end{enumerate}

Has the following properties:
\begin{enumerate}

\item
$\Phi\circ\Phi(\theta) = \Phi(\theta)$

\item
$\forall \theta_*\in[\theta]$, $\Phi(\theta_*) = \Phi(\theta_0)$ where $\theta_0$ representative of $[\theta]$

\end{enumerate}

\item
A symmetry broken posterior distribution is a posterior distribution that has the (unnormalized) form
\[
\prod_kF_k(\Phi_{\mathcal{F}_M}\circ ... \circ \Phi_{\mathcal{F}_1}(\theta))
\]
where $\mathcal{F}_1,...,\mathcal{F}_M$ have equivalence relations $\Xi_1,...,\Xi_M$ corresponding to different types of symmetries equipped, respectively.

\end{itemize}


\section{Breaking symmetries post-inference}

\begin{itemize}

\item[Context:]
Have a posterior distribution $\prod_kF_k(\theta)$ where inference has been performed ($\theta$ has/can be identified)

\item[Problem:]
Nonidentifiability. There exists a (local) symmetry $\sigma$ s.t. $\prod_kF_k(\theta) \propto \prod_kF_k(\sigma(\theta))$ or for local, $F_k(\theta)\propto F_k(\sigma(\theta))$

\item[When:]
Inference algorithm performance is not an issue in the presence of a symmetry.

\item[Solution:]
Use the symmetry broken posterior distribution

\end{itemize}


\subsection{Scaling symmetries}

Scaling symmety is a symmetry $\sigma$ which multiplies $\theta$ pointwise by a vector
\[
v=(r_1,...,r_N) = (e^{d_1},...,e^{d_N})
\]
where $r_n\in \mathbb{R}_+$ and $d_n=\log r_n$.

\begin{itemize}

\item
Let $\Sigma_v$ be the diagonal matrix with $v$ on the diagonal. Then $\sigma(\theta)=\Sigma_v\theta$.

\item
Let $d=(d_1,...,d_N)$. Let $v_d = (e^{d_1},...,e^{d_N})$. Then the scaling symmetries of the model are $\sigma_{d}(\theta)=\Sigma_{v_d}\theta$, $d\in\mathcal{N}(C)$.

\item
Then $\{\Sigma_{v_d}\theta:d\in\mathcal{N}(C)\}\subset\Theta$ is an equivalence class for the model $(\theta,F)$.

\end{itemize}

\section{Breaking symmetries pre-inference}

\begin{itemize}

\item
Idea: add factors with constraints to make nullspace the 0 vector

How can this be done in a structured way?

Start with original nullspace and determine what constraint to add?

What factors do these constraints correspond to?

Can these factors be added without modifying the likelihood?


\end{itemize}


\section{Notes}

\begin{itemize}

\item
Class of transformations of model parameters $T\subset\{f:\Theta\rightarrow\Theta\}$. Subset of local symmetries contained in $T$ $S_T\subset T$ defined by
\[
S_T=\{\sigma\in T|\sigma \text{ is a local symmetry }\} = \bigcap_k\{\sigma\in T | F_k(\theta)\propto F_k\left(\sigma(\theta)\right)\}
\]

\item
\href{https://en.wikipedia.org/wiki/Equivalence_class}{Equivalence class} definition

A factor graph (model?) with equivalence relation $\Xi$, denoted $\mathcal{F}=(\theta,F,\Xi)$, induces a partition of $\Theta$ into equivalence classes, with the equivalence relation $\Xi$ that there exists a symmetry between any two members of an equivalence class. $\Xi$ specifies the type of symmetry.

Classes may be singletons if the point is symmetric only to itself.

Denote $[\theta]$ to be the equivalence class of $\theta$.

Denote $\Theta/\Xi$ to be the set of all equivalence classes in $\Theta$ w.r.t $\Xi$, called the quotient set of $\Theta$ by $\Xi$.

\item
A symmetry detector $\Delta_\mathcal{F}:\Theta\rightarrow\Theta/\Xi$ is a measurable? function that maps $\theta$ to $[\theta]$ w.r.t. $\Xi$.

Symmetry detection is identifying the equivalence class?

\item
Definition 1: A symmetry breaker $\phi_\mathcal{F}:\Theta/\Xi\rightarrow\Theta$ is a measurable function that maps $[\theta]$ to $\theta_0$ w.r.t. $\Xi$ where $\theta_0$ is a representative of $[\theta]$.

Definition 2: A symmetry breaker $\phi_\mathcal{F}:\Theta/\Xi\rightarrow\Theta$ is a measurable function that maps $[\theta]$ to $\theta\in[\theta]$ w.r.t. $\Xi$.

\item
An automatic symmetry breaker $\Phi_\mathcal{F}:\Theta\rightarrow\Theta$ is the composition $\Phi_\mathcal{F}=\phi_\mathcal{F}\circ\Delta_\mathcal{F}$ of a symmetry breaker with the corresponding symmetry detector.

A symmetry broken posterior distribution is a posterior distribution that has the (unnormalized) form
\[
\prod_kF_k(\Phi_{\mathcal{F}_M}\circ ... \circ \Phi_{\mathcal{F}_1}(\theta))
\]
where $\mathcal{F}_1,...,\mathcal{F}_M$ have equivalence relations $\Xi_1,...,\Xi_M$ corresponding to different types of symmetries equipped, respectively.

$\Phi$ is not a symmetry (no measurable inverse)

\item
Scaling symmety is a symmetry $\sigma$ which multiplies $\theta$ pointwise by a vector
\[
v=(r_1,...,r_N) = (e^{d_1},...,e^{d_N})
\]
where $r_n\in \mathbb{R}_+$ and $d_n=\log r_n$.

\begin{itemize}

\item
Case of $r_n\in\mathbb{R}_-$ is covered by a combination of scaling symmetry and sign-flip symmetry.

\item
Let $\Sigma_v$ be the diagonal matrix with $v$ on the diagonal. Then $\sigma(\theta)=\Sigma_v\theta$.

\item
Let matrix $C$ be the matrix of constraints on $d_n$. The scaling symmetries of the model are the vectors in the null space of $C$, $\mathcal{N}(C)$.

\item
Let $d=(d_1,...,d_N)$. Let $v_d = (e^{d_1},...,e^{d_N})$. Then the scaling symmetries of the model are $\sigma_{d}(\theta)=\Sigma_{v_d}\theta$, $d\in\mathcal{N}(C)$.

\item
Then $\{\Sigma_{v_d}\theta:d\in\mathcal{N}(C)\}\subset\Theta$ is an equivalence class for the model $(\theta,F)$.

\item
The symmetry detector $\Delta_{\mathcal{F}_S}$ for a scaling symmetry is defined as
\[
\Delta_{\mathcal{F}_S}(\theta) = \{\Sigma_{v_d}\theta: d\in\mathcal{N}(C)\}
\]

\item A symmetry breaker (definition ) for a scaling symmetry is
\[
\phi_{\mathcal{F}_S}([\theta]) =
\begin{cases}
\theta & \mathcal{N}(C) = \{\vec{0}\} \\
\Sigma_{v_d}\theta, \; \|v_d\|=1 & \mathcal{N}(C) \neq \{\vec{0}\}
\end{cases}
\]

\item
A symmetry breaker (definition 1) for a scaling symmetry is
\[
\phi_{\mathcal{F}_S}([\theta]) = \Sigma_d\theta
\]

Let $\theta_* = \Sigma_{d^*}\theta$ for some $d^*\in\mathcal{N}(C)$.

Then
\begin{align*}
\Delta_\Xi(\theta_*) &= \{\Sigma_d\theta_*:d\in\mathcal{N}(C)\} \\
&= \{\Sigma_d\Sigma_{d^*}\theta:d\in\mathcal{N}(C)\} \\
&= \{\Sigma_{d+d^*}\theta:d\in\mathcal{N}(C)\} \\
&= \{\Sigma_d\theta:d\in\mathcal{N}(C)\}
\end{align*}
where the above follows as $d+d^*\in\mathcal{N}(C)$. Hence $\Delta_\Xi$ maps members of an equivalence class to the equivalence class.

Symmetry breaker: need to constrain $d\in\mathcal{N}(C)$ to get a unique representative? How to remove dependence on $\theta$?

\item
G\&H 5.3: latent variable $z_i=X_i\beta+\epsilon_i$, $\epsilon_i\sim N(0,\sigma^2)$, scaling of $\beta$ doesn't change sign of $z_i$. Proposed solution is just to fix $\sigma$.

\item
N'98: same problem as above but now covariance matrix. Proposed solution is just to fix $\Sigma_{1,1}=1$.

\end{itemize}

\end{itemize}




\end{document}