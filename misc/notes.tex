\documentclass{article}


\usepackage{amsmath}
\usepackage{amssymb}
\usepackage{hyperref}


\begin{document}

\begin{itemize}

\item
Symmetry $\sigma:\Theta\rightarrow\Theta$ is a measurable function with a measurable inverse satisfying
\[
\prod_kF_k(\theta)\propto\prod_kF_k\left(\sigma(\theta)\right)
\]

\item
Local symmetry $\sigma$ satisfies
\[
F_k(\theta) \propto F_k\left(\sigma(\theta)\right)
\]

\item
Class of transformations of model parameters $T\subset\{f:\Theta\rightarrow\Theta\}$. Subset of local symmetries contained in $T$ $S_T\subset T$ defined by
\[
S_T=\{\sigma\in T|\sigma \text{ is a local symmetry }\} = \bigcap_k\{\sigma\in T | F_k(\theta)\propto F_k\left(\sigma(\theta)\right)\}
\]

\item
\href{https://en.wikipedia.org/wiki/Equivalence_class}{Equivalence class} definition

A factor graph (model?) with equivalence relation $\Xi$, denoted $\mathcal{F}=(\theta,F,\Xi)$, induces a partition of $\Theta$ into equivalence classes, with the equivalence relation $\Xi$ that there exists a symmetry between any two members of an equivalence class. $\Xi$ specifies the type of symmetry.

Classes may be singletons if the point is symmetric only to itself.

Denote $[\theta]$ to be the equivalence class of $\theta$.

Denote $\Theta/\Xi$ to be the set of all equivalence classes in $\Theta$ w.r.t $\Xi$, called the quotient set of $\Theta$ by $\Xi$.

\item
A symmetry detector $\Delta_\mathcal{F}:\Theta\rightarrow\Theta/\Xi$ is a measurable? function that maps $\theta$ to $[\theta]$ w.r.t. $\Xi$.

\item
Definition 1: A symmetry breaker $\phi_\mathcal{F}:\Theta/\Xi\rightarrow\Theta$ is a measurable function that maps $[\theta]$ to $\theta_0$ w.r.t. $\Xi$ where $\theta_0$ is a representative of $[\theta]$.

Definition 2: A symmetry breaker $\phi_\mathcal{F}:\Theta/\Xi\rightarrow\Theta$ is a measurable function that maps $[\theta]$ to $\theta\in[\theta]$ w.r.t. $\Xi$.

\item
An automatic symmetry breaker $\Phi_\mathcal{F}:\Theta\rightarrow\Theta$ is the composition $\Phi_\mathcal{F}=\phi_\mathcal{F}\circ\Delta_\mathcal{F}$ of a symmetry breaker with the corresponding symmetry detector.

A symmetry broken posterior distribution is a posterior distribution that has the (unnormalized) form
\[
\prod_kF_k(\Phi_{\mathcal{F}_M}\circ ... \circ \Phi_{\mathcal{F}_1}(\theta))
\]
where $\mathcal{F}_1,...,\mathcal{F}_M$ have equivalence relations $\Xi_1,...,\Xi_M$ corresponding to different types of symmetries equipped, respectively.

\item
Scaling symmety is a symmetry $\sigma$ which multiplies $\theta$ pointwise by a vector
\[
v=(r_1,...,r_N) = (e^{d_1},...,e^{d_N})
\]
where $r_n\in \mathbb{R}_+$ and $d_n=\log r_n$.

\begin{itemize}

\item
Case of $r_n\in\mathbb{R}_-$ is covered by a combination of scaling symmetry and sign-flip symmetry.

\item
Let $\Sigma_v$ be the diagonal matrix with $v$ on the diagonal. Then $\sigma(\theta)=\Sigma_v\theta$.

\item
Let matrix $C$ be the matrix of constraints on $d_n$. The scaling symmetries of the model are the vectors in the null space of $C$, $\mathcal{N}(C)$.

\item
Let $d=(d_1,...,d_N)$. Let $v_d = (e^{d_1},...,e^{d_N})$. Then the scaling symmetries of the model are $\sigma_{d}(\theta)=\Sigma_{v_d}\theta$, $d\in\mathcal{N}(C)$.

\item
Then $\{\Sigma_{v_d}\theta:d\in\mathcal{N}(C)\}\subset\Theta$ is an equivalence class for the model $(\theta,F)$.

\item
The symmetry detector $\Delta_{\mathcal{F}_S}$ for a scaling symmetry is defined as
\[
\Delta_{\mathcal{F}_S}(\theta) = \{\Sigma_{v_d}\theta: d\in\mathcal{N}(C)\}
\]

\item A symmetry breaker (definition 2) for a scaling symmetry is
\[
\phi_{\mathcal{F}_S}([\theta]) =
\begin{cases}
\theta & \mathcal{N}(C) = \{\vec{0}\} \\
\Sigma_{v_d}\theta, \; \|v_d\|=1 & \mathcal{N}(C) \neq \{\vec{0}\}
\end{cases}
\]

\item
G\&H 5.3: latent variable $z_i=X_i\beta+\epsilon_i$, $\epsilon_i\sim N(0,\sigma^2)$, scaling of $\beta$ doesn't change sign of $z_i$. Proposed solution is just to fix $\sigma$.

N'98: same problem as above but now covariance matrix. Proposed solution is just to fix $\Sigma_{1,1}=1$.

\end{itemize}

\end{itemize}




\end{document}