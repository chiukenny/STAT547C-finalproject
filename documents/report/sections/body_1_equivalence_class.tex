% !TEX root = ../main.tex

% Body 1 section

\section{Equivalence class formulation}

The first formulation of symmetry detection and breaking that we introduce is based on the concept of equivalence classes. For a given factor graph, symmetry detection can be framed as a problem of finding the set of variables that are symmetric (an equivalence class). Symmetry breaking can then be framed as choosing a single member (a representative) from the set of symmetric variables based on some desired criteria. We formalize these ideas mathematically.

\begin{defn}
Consider a factor graph $(\theta,F)$. Define $\Xi$ to be the equivalence relation such that for $\theta,\theta^*\in\Theta$, we have $\theta\Xequiv\theta^*$ if and only if there exists a local symmetry $\sigma$ such that $\sigma(\theta)=\theta^*$.
\end{defn}

\noindent We show that $\Xi$ is a proper equivalence relation.

\begin{proof}
Consider $\theta_1, \theta_2, \theta_3\in\Theta$.
\begin{enumerate}

\item
\textbf{reflexivity}: let $\sigma$ be the identity map. Then $F_k(\theta_1) = F_k(\sigma(\theta_1))$ for all $k$ and so $\theta_1\Xequiv\theta_1$.

\item
\textbf{symmetry}: suppose $\theta_1\Xequiv\theta_2$. Then there exists some local symmetry $\sigma$ such that $\sigma(\theta_1)=\theta_2$ and $F_k(\theta_1) \propto F_k(\theta_2)$ for all non-prior $F_k$. By definition of symmetry, $\sigma^{-1}$ exists and $\sigma^{-1}(\theta_2)=\theta_1$. Then
\[
F_k(\theta_2) \propto F_k(\theta_1) = F_k(\sigma^{-1}(\theta_2))
\]
for all non-prior $F_k$ and hence $\theta_2\Xequiv\theta_1$.

\item
\textbf{transitivity}: suppose $\theta_1\Xequiv\theta_2$ and $\theta_2\Xequiv\theta_3$. Then there exists local symmetries $\sigma_1$, $\sigma_2$ such that
\begin{align*}
\sigma_1(\theta_1)&=\theta_2 & F_k(\theta_1)&\propto F_k(\theta_2) \\
\sigma_2(\theta_2)&=\theta_3 & F_k(\theta_2)&\propto F_k(\theta_3)
\end{align*}
for all non-prior $F_k$. Let $\sigma_3=\sigma_2\circ\sigma_1$ and so $\sigma_3$ is also measurable with a measurable inverse. Then
\[
\sigma_3(\theta_1)=\sigma_2(\sigma_1(\theta_1)) = \theta_3
\]
and
\[
F_k(\theta_1) \propto F_k(\theta_2) \propto F_k(\theta_3) = F_k(\sigma_3(\theta_1))
\]
for all non-prior $F_k$. Hence $\theta_1\Xequiv\theta_3$.

\end{enumerate}
\end{proof}

Notice that $\Xi$ is only well-defined in the context of a factor graph $(\theta,F)$. It is therefore mathematically more convenient to consider factor graphs equipped with a $\Xi$ that corresponds to a specific type of symmetry. We denote this as $\mathcal{F}=(\theta,F,\Xi)$. The constraint matrix $\mathcal{C}$ of $\mathcal{F}$ described in Section \ref{sec:scaling} is constructed from the constraints specified by $\Xi$ on each factor $F_k$.  We now formally define symmetry detection and symmetry breaking in terms of functions on equivalence classes.

\begin{defn}
A \textit{symmetry detector} $\Delta_\mathcal{F}:\Theta\rightarrow\Theta/\Xi$ is a measurable function that maps $\theta$ to $[\theta]$ with respect to $\Xi$.
\end{defn}

\begin{defn}
A \textit{symmetry breaker} $\phi_\mathcal{F}:\Theta/\Xi\rightarrow\Theta$ is a measurable section that maps $[\theta]$ to a representative $\theta^*\in[\theta]$.
\end{defn}

\begin{defn}
An \textit{automatic symmetry breaker} $\Phi_\mathcal{F}=\phi_\mathcal{F}\circ\Delta_\mathcal{F}:\Theta\rightarrow\Theta$ is the composition of a symmetry breaker and its corresponding symmetry detector. It maps $\theta$ to a representative $\theta^*$ of its equivalence class.
\end{defn}

The dependence of $\Delta$, $\phi$, and $\Phi$ on the factor graph $\mathcal{F}$ is made clear by the subscript. When considering two functions of the same type, e.g. $\Phi_{\mathcal{F}_1}$ and $\Phi_{\mathcal{F}_2}$, we will view them as being defined on the same factor graph $(\theta,F)$ but with $\Xi_1$ and $\Xi_2$ corresponding to different types of symmetries.

This formulation is most useful when the main concern of having symmetries is parameter nonidentifiability. If inference for a posterior distribution was performed but it is unclear if the estimated value is ``correct", it may instead be more interpretable to work with the symmetry-broken posterior distribution
\[
\prod_kF_k(\Phi_{\mathcal{F}_M}\circ ...\circ\Phi_{\mathcal{F}_1}(\theta))
\]
where $\Phi_{\mathcal{F}_1},...,\Phi_{\mathcal{F}_M}$ correspond to $M$ different types of symmetries that are to be broken. The corresponding $\phi_{\mathcal{F}_1},...,\phi_{\mathcal{F}_M}$ are chosen to select representatives from the respective equivalence classes based on some desired criteria.

We now show how detection and breaking of scaling and permutation symmetries are described under the equivalence class formulation.


\subsection{Scaling symmetries}

We define scaling symmetry detection and breaking under the equivalence class formulation.

\begin{defn}
Let $\Sigma_d$ be the diagonal matrix with $(e^{d_1},...,e^{d_N})$ on the diagonal where $d_n\in\mathbb{R}$. A scaling symmetry $\sigma_d$ can be written as
\[
\sigma_d(\theta)=\Sigma_d\theta
\]
where the right-hand side is taken as a matrix-vector product. The matrix $\Sigma_d$ is positive definite and so the inverse exists as required by the definition of symmetry. We then define the \textit{scaling symmetry detector} to be
\[
\Delta_\mathcal{F}(\theta) = \{\Sigma_d\theta:d\in\mathcal{N}(\mathcal{C})\}=[\theta]
\]
where $\mathcal{C}$ is the constraint matrix of $\mathcal{F}$.
\end{defn}

\noindent We show that this detector correctly maps the variables $\theta$ to their equivalence classes $[\theta]$ with respect to $\Xi$. That is, the detector maps two variables $\theta$, $\theta^*\in\Theta$ to the same equivalence class if and only if there exists a scaling symmetry $\sigma_d$ such that $\sigma_d(\theta)=\theta^*$.

\begin{proof}
\textit{Sufficiency}: Suppose that $\theta^*=\Sigma_{d^*}\theta$ for some $d^*\in\mathcal{N}(\mathcal{C})$. Then
\begin{align*}
\Delta_\mathcal{F}(\theta^*) &= \{\Sigma_d\theta^*:d\in\mathcal{N}(\mathcal{C})\} \\
&= \{\Sigma_d\Sigma_{d^*}\theta:d\in\mathcal{N}(\mathcal{C})\} \\
&= \{\Sigma_{d+d^*}\theta:d\in\mathcal{N}(\mathcal{C})\} \\
&= \{\Sigma_d\theta:d\in\mathcal{N}(\mathcal{C})\} \\
&= \Delta_\mathcal{F}(\theta)
\end{align*}
where the above follows because $\Sigma_d$ and $\Sigma_{d^*}$ are diagonal and $d+d^*\in\mathcal{N}(\mathcal{C})$.
\\

\noindent \textit{Necessity}: We show this by the contrapositive. Suppose that $\theta^*\neq\Sigma_{d}\theta$ for all $d\in\mathcal{N}(\mathcal{C})$. Then $\theta^*\notin\Delta_\mathcal{F}(\theta)$ by definition of the symmetry detector. Also, by definition of equivalence class, $\theta^*\in[\theta^*]=\Delta_\mathcal{F}(\theta^*)$. Hence $\Delta_\mathcal{F}(\theta)\neq\Delta_\mathcal{F}(\theta^*)$ and so the detector maps $\theta$, $\theta^*$ to different equivalence classes.
\end{proof}

It is difficult to provide a concrete example of a \textit{scaling symmetry breaker} $\phi_\mathcal{F}$ for two reasons. The first reason is that the equivalence class is notationally described by using the input $\theta$ as a reference point. If $[\theta_1]$ and $[\theta_2]$ describe the same equivalence class, the breaker must map both to the same representative. The second reason is that the equivalence class is nonlinear in the space of matrices $\Sigma_d$ (it is described by the linear space $\mathcal{N}(\mathcal{C})$ rather). This means that for a given $\theta$, operations such as scaling may not return a member of the same equivalence class. These two reasons make it challenging to mathematically define a scaling symmetry breaker that correctly extracts the representative for a given equivalence class. An example of a possible breaker is the one that returns the minimum norm, i.e.,
\[
\phi_\mathcal{F}([\theta]) = \underset{\theta\in[\theta]}{\arg\min}\;\|\theta\|
\]
However, this mathematically convenient notation hides the question of how to computationally solve for the minimum norm (if it exists) when $[\theta]$ may be an infinite nonlinear space. Consideration will also need to be given in how to deal with multiple members having the same norm.


\subsection{Permutation symmetries}

We define permutation symmetry detection and breaking under the equivalence class formulation. As permutation symmetries are not local, the definition of $\Xi$ is modified to consider the existence of a non-local symmetry between two members of an equivalence class in this context.

\begin{defn}
Let $\Pi$ be a permutation matrix (a matrix obtained from permuting rows of the identity matrix). A permutation symmetry $\sigma$ can be written as
\[
\sigma(\theta)=\Pi\theta
\]
where again the right-hand side is taken to be a matrix-vector product. Permutation matrices are orthogonal and so the inverse exists and is given by $\Pi^T$. We say that $\Pi\in\mathcal{F}$ if the rows that correspond to non-permutable variables have 1 on the diagonal. The \textit{permutation symmetry detector} is then defined to be
\[
\Delta_\mathcal{F}(\theta) = \left\{\Pi\theta:\Pi\in\mathcal{F}\right\} = [\theta]
\]
Under this definition, it is assumed that there are no restrictions on the permutations of the permutable variables. If variables are permutable with only certain others, multiple permutation symmetry detectors are needed where each detects the symmetries among a subset of permutable variables.
\end{defn}

Notice that if $\Pi,\Pi^*\in\mathcal{F}$, then the matrix product $\Pi\Pi^*$ is itself a permutation matrix by properties of permutation matrices. Furthermore, $\Pi\Pi^*\in\mathcal{F}$ as both matrices have 1 on the diagonal in non-permutable rows and thus so does the product. We use this fact to again show that $\Delta_\mathcal{F}(\theta)=\Delta_\mathcal{F}(\theta^*)$ if and only if there exists a permutation symmetry $\sigma$ such that $\sigma(\theta)=\theta^*$.

\begin{proof}
\textit{Sufficiency}: Suppose that $\theta^*=\Pi^*\theta$ for some $\Pi^*\in\mathcal{F}$. Then
\begin{align*}
\Delta_\mathcal{F}(\theta^*) &= \left\{\Pi\theta^*:\Pi\in\mathcal{F}\right\} \\
&= \left\{\Pi\Pi^*\theta:\Pi\in\mathcal{F}\right\} \\
&= \left\{\Pi\theta:\Pi\in\mathcal{F}\right\} \\
&= \Delta_\mathcal{F}(\theta)
\end{align*}
where the above follows from the aforementioned fact.
\\

\noindent \textit{Necessity}: The proof for the converse follows the same argument as for the scaling symmetry. Hence this permutation symmetry detector correctly maps the variables to their equivalent classes.
\end{proof}

An example of a \textit{permutation symmetry breaker} is easy to describe. For simplicity, suppose that the permutable variables in $\theta$ have distinct values. Then an example permutation breaker $\phi_\mathcal{F}$ is one such that $\phi_\mathcal{F}([\theta])=\theta^*$ where if $\theta_i$ and $\theta_j$ are permutable, then $\theta^*_i<\theta^*_j$. In the context of probabilistic programming, this reduces to a sorting problem involving certain entries of $\theta$.

One distinction between permutation equivalence classes and scaling equivalence classes is that the permutation equivalence classes are always finite. A class can have at most $N!$ members corresponding to the possible permutations of the variables. This makes the permutation symmetry breaker computationally easier to implement compared to the scaling symmetry breaker in that in the worst case, the breaker can just iterate over the equivalence class to pick a representative.


\subsection{Other symmetries}

We end our discussion of the equivalence class formulation with a brief comment about the sign-flip and translation symmetries. The formulation for the sign-flip symmetry would likely closely resemble that of the scaling symmetry where the symmetries are given by the null space of the constraint matrix. An added benefit is that the space of symmetries is also finite due to having only two possible values for each exponent. On the other hand, the translation symmetry appears to be difficult to describe under the equivalence class formulation. The dependence on $\theta$ in the symmetry itself makes it challenging to define detection as a proper function.


% ...